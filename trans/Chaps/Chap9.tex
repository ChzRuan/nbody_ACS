%!TEX root = ../CallenThermo.tex
% - 翻译:CuI
\chapter{一级相变}
\label{chap9}

\section{能量最小原理}
\label{sec9.1}

通常情况下在室温和大气压的条件下水是液体,
但是如果被冷却到$273.15K$以下就会凝固;
而如果被加热到$373.15K$以上就会汽化。
在这些温度下物质的性质都会发生急剧的变化——也就是“相变”。
在高压下水会经历一些额外的从一种固态到另一种固态的相变过程。
这些不同的固相,被定为“冰I”,“冰II”,“冰III”,……,在晶体结构以及基本所有热力学性质上都不同(比如压缩系数,摩尔热容,以及各种摩尔势比如$u$或者$f$)。
水的“相图”如图\ref{fig9.1}所示。

每一个相变都对应着热力学基本关系的一个线性区域(比如图\ref{fig8.2}的$BHF$),
并且每一个都可以看成潜在的基本关系中稳定条件(凸的或者凹的)失效的结果。

在本节中我们要考虑基本关系不稳定的系统。
通过定性考虑这种系统的涨落我们会看到{\it 涨落会受到潜在的基本关系的细节的深刻影响}。
与此相对的,{\it 广延量的均值只能反应稳定的热力学基本关系。}

关于潜在的基本关系的形式对于热力学涨落的影响方式的考虑会给
第\ref{chap8}章中稳定性的考虑以及图\ref{fig8.2}的构造(其中热力学基本关系是通过切平面的包络构造出来的)提供物理解释。

一个简单的力学模型可以通过一个直观易懂的类比来解释这种考虑。
考虑一段半圆形的两端封闭的管子。
这个管子如同倒着的$U$型垂直竖立在桌子上(图\ref{fig9.2})。
这个管子内包含一个可以自由移动的活塞将其分成两部分,
每一部分都包含一摩尔的气体。
这个系统的对称性会被证明有重要的效果,
而为了破坏这个对称性我们考虑管子的每一段含有一个小的金属“滚珠”(也就是一个小的金属球)。
两个金属球是热膨胀系数不同的两种金属构成的。

在一些特别的温度下,比如我们记为$T_c$,两个金属球的半径相同;
而在温度高于$T_c$的时候,右边的球更大。

活塞暂时被放在管子的顶端,可以落入两条腿中的任意一个,
从而压缩那一条腿中的气体,而另一条腿中的气体则会膨胀。
在这两种互斥的平衡态中,压强差都正好被活塞质量的效果补偿。

如果没有这两个滚珠的话这两个互斥的平衡态是完全等价的。
但当存在滚珠的时候,如果$T>T_c$那么落入左边是一个更加稳定的平衡态,
如果$T<T_c$那么落入右边就成了更加稳定的平衡态。

从热力学的观点来看,系统的 Helmholtz 势是$F=U-TS$,
而能量$U$包含了活塞的重力势能以及我们所熟悉的两部分气体的热力学能量
(当然也包括两个滚珠的内能,不过我们假定这部分很小而且两者相同)。
从而系统的Helmholtz势就有两个局域极小值,
而更低的哪一个极小值对应着活塞落入包含的球更下的那一侧。

活塞从一边到另一边的平衡位置之间的移动也可以通过在给定温度下倾斜桌子达到类似的效果——或者在类似的热力学中,通过加入一些温度以外的热力学参数实现。

平衡态从一个局域极小移动到另一个构成了{\it 一级相变},
这是由于温度或者其它热力学参数的改变导致的。

{\it 一级相变联系的两个态是有区别的,它们出现在热力学构型空间中独立的区域。}

为了讨论“临界现象”和“二级相变”(第\ref{chap10}章),
暂时考虑滚珠全同或者不存在的情况会很有用。
于是在低温时两个互斥的极小值是等价的。
不过随着温度的升高,活塞的两个平衡位置会在管子中上升接近顶端。
超过一个特别的温度$T_{cr}$以后,就只有活塞位于管子的顶端这一个平衡位置了。
相反地,如果把温度从$T>T_{cr}$降到$T<T_{cr}$,
这唯一的平衡态会分成两个(对称的)平衡态。
这个温度$T_{cr}$就是“临界温度”而在$T_{cr}$的变化就是“二级相变”。

{\it 二级相变联系的两个态在热力学构型空间中是连续的。}

在这一章中我们考虑一级相变。
二级相变会在第\ref{chap10}章中讨论。
在那里我们也会对这个“力学模型”做定量地考虑,尽管在这里我们仅仅做了定性的讨论。

回到两个球不同的情况,考虑活塞位于更大的极小值——
也就是跟更大的滚珠位于管子的同一侧。
处于Helmholtz势那样的一个极小值时,
即使经历热涨落(“布朗运动”)活塞也会暂时保持在那个极小值处。
在足够场的时间后一个巨大的涨落会带动活塞“越过顶端”并达到稳定的极小值。
之后它就会呆在这个更深的极小值处知道一个更大(也更加不可能)的扰动把它待会次稳定的极小值处,然后这整个过程会反复重现。
随着幅度的增大涨落的概率会下降得非常快(我们会在第\ref{chap19}章看到这一点)
以至于{\it 在大部分时间中系统会位于更加稳定的极小值处。}
宏观热力学中多有这些动力学都被忽略了,因为它只关心稳定的平衡态。

为了在更加偏向热力学的环境中讨论相变的动力学,
把我们的注意力转移到更加熟悉的热力学系统会更加方便,
它的热力学势同样具有两个局域极小值并且被一段凹的不稳定区域分开。
特别地我们考虑一个容器中压力为一个大气压而且温度高于$373.15K$的水蒸气(也就是高于水的“常规沸点”)。
我们把注意力集中到一个小的子系统——一个半径大小(可以改变)使得任意时刻都包含有一毫克水的球形区域。
这个子系统等价于跟一个热库和一个蓄压器接触,
而平衡条件是子系统的Gibbs势$G(T,P,N)$处于极小值。
由平衡条件决定的两个独立变量是子系统的能量$U$和体积$V$。

如果Gibbs势形如图\ref{fig9.3}所示那样,
其中$X_j$是体积,那么系统在更低的极小值处是稳定的。
这个极小值对应着比第二局域极小值更大的体积(或者更小的密度)。

考虑体积涨落的行为。
这种涨落是自发而且连续发生的。
图\ref{fig9.3}中曲线的斜率表征了一个强度量(在当前情况下与压强不同),
它扮演着符合Le Chatelier 原理使系统密度趋于均匀的回复“力”的角色。
偶尔涨落会太大以至于使得系统越过极大值,达到第二个极大值的区域。
接下来系统就会呆在第二个极小值区域——不过只会呆一小会儿。
要跨过第二个极小值处更低的势垒,一个更小(因此也就更加常见)的涨落就足够了。
系统很快就回到了它的稳定态。
因此很小的高密度液滴(液相!)偶尔会在气体中形成,存在一小会儿然后消失。

如果第二个极小值距离最小值很远,而且它们之间的势垒很高,
那么从一个到另一个的涨落就会变得非常不可能。
在第\ref{chap19}章中会说明,这类涨落的概率会随着中间自由能的势垒的高度指数下降。
在固态系统(其中相互作用能量比较高)中由于中间势垒高得让从一个极小值变到另一个的时间的量级超过了宇宙的年龄,所以通常不会有多个极小值。
陷入这种“亚稳态”极小值的系统{\it 实际上}处于稳定平衡态(就像更深的极小值并不存在一样)。

回到水蒸气处于高于“沸点”的某个温度的情形,假设我们降低整个系统的温度。
Gibbs势的变化形式如同图\ref{fig9.4}中显示的那样。
在温度$T_4$两个极小值会相等,而在这个温度以下,高密度(液体)相会变成绝对稳定的。
因此$T_4$就是相变的温度(在给定的压力下)。

如果蒸汽从相变温度缓慢冷却,系统会发现之前所处的绝对稳态现在变成了亚稳态。
迟早系统的一个涨落就会“发现”真正的稳态,形成液体凝结核。
这个核会迅速变大,然后整个系统会突然经历相变。
实际上系统通过一个“试探”涨落发现更稳定的态所需的时间在蒸汽到液体凝结的情况下短到难以观察。
不过在从液体到冰的持续时间在一个纯粹的情况下很容易观察到。
被冷却到低于凝固点(结冰)温度的液体被称为是“过冷的”。
轻轻敲一下容器,产生的纵波就会导致多个“密集”和“稀疏”的区域,
而这些外部诱导的涨落会代替自发的涨落引发剧烈的相变。

当把Gibbs势的极小处的值相对温度画出来的时候会引出一个有用的视角。
这个结果如图\ref{fig9.5}所示。
如果从图\ref{fig9.4}中选取极小值那么久会只有两个这种曲线,
不过任意数值都是可能的。
在平衡态最小的极小值是稳定的,所以真正的Gibbs势是图\ref{fig9.5}中的曲线的下包络。
熵的不连续(也就是潜热)对应着包络函数斜率的不连续。

图\ref{fig9.5}需要增加一个额外的维度,增加的坐标$P$作用类似于$T$。
于是Gibbs势就被表示成下包络曲面,和三个单个的相面重合。
这些曲面之间相交曲线在$P-T$平面上的投影就是我们熟悉的相图(比如图\ref{fig9.1})。

相变发生在系统从一个包络曲面越过相交线到另一个包络曲面时所处的态。

图\ref{fig9.4}的变量$X_j$或者$V$可以是任意广延量。
在顺磁到铁磁的相变中,$X_j$是磁矩。
在从一种晶体形式到另一种晶体形式的相变中(比如从立方体到六边形)
相应的参数$X_j$是一个晶体对称性变量。
在溶解度的相变中则可能是一种组分的摩尔数。
随后我们会看到这类相变的例子。
这些都符合前面描述的一般模式。

在一级相变中两个相的摩尔Gibbs势是相等的,但是其它摩尔势(比如$u,t,f$)
在相变前后是不连续的,摩尔体积和摩尔熵也是如此。
两种相占据了“热力学空间”的不同区域,而除了Gibbs势以外任何性质的相等都只是巧合。
摩尔势的不连续性就是定义一级相变的性质。

如图\ref{fig9.6}所示,如果沿气液共存曲线远离固相(也就是向着更高的温度),
摩尔体积和摩尔能量的间断会逐渐变小。
两种相会变得越来越相似。
最再气液共存曲线的终点,两种相会变得不可分辨。
一级相变会退化成一个更加微妙的相变,也就是我们将会在第\ref{chap10}章中介绍的{\it 二级相变}。
共存曲线的终点被称为{\it 临界点}。

临界点的存在阻碍了我们明确区分通称为{\it 气体}和通称为{\it 液体}的可能性。
在通过一级相变穿过气液共存曲线的时候,我们区分了两种态,
一个“明确”为气体而另一个“明确”为液体。
但从其中一个开始(比如液体,处于共存曲线上方)
我们可以沿着任意一条绕过临界点的路径到达另一个态(“气”态)而不经历相变!
从而通称的{\it 气体}和{\it 液体}相较于严格定义的记法更多是直觉上的含义。
液体和气体共同组成了{\it 流体相}。
除此以外我们在气液一级相变中对于“液相”和“气相”应该遵循标准的用法。

在图\ref{fig9.1}中有另一个非常有趣的点:气液共存曲线的另一个终点。
这个点是三条共存曲线的交点,它也是唯一一个气相液相和固相共存的点。
这种三相相容的点被称为“三相点”——这里是水的三相点。
被唯一确定的水的三相点的温度被定义(可以是任意的)为
Kelvin温标下的$273.16K$这一数值(在第\ref{sec2.6}节中已经介绍过)。

\section{熵的不连续——潜热}
\label{sec9.2}

类似图\ref{fig9.1}的相图被共存曲线分为不同的使得某一种相是稳定态的区域。
这些曲线上的每一点两种相都具有完全一样的摩尔Gibbs势,因此两种相可以共存。
考虑水处于图\ref{fig9.1}a中“冰”这一区域的某个温度和压强下。
为了增加冰的温度,每使温度升高一Kelvin我们都要提供大概$2.1kJ/kg$的热量
(冰的比热容)。
如果用固定的速度提供热量,温度也会以几乎固定的速度增加。
但当温度到达气液共存线上的"熔化温度",温度就不再增加了。
如果继续供热,冰就会熔化形成同样温度的水。
熔化每$kg$的冰需要$335kJ$的热量。
在系统到达共存曲线后(也就是处于熔点温度),
任意时刻容器中水的多少都决定于进入容器的热量的多少。
当最终供给足够多的热量的时候,冰会完全熔化,继续加热会再次导致温度的升高——
而此时的速率是由水的比热容决定的($\simeq 4.2kJ/kg-K$).

一摩尔固体熔化需要的热量就是{\it 融化热}(或者叫{\it 熔化潜热})。
它和气相与固相的摩尔熵的差相联系。
\begin{equation}
\label{equ9.1}
\ell_{LS}=T[s^{(L)}-s^{(S)}]
\end{equation}
其中$T$是给定压强下熔点的温度。

更一般地,任何一级相变的潜热是
\begin{equation}
\label{equ9.2}
\ell=T\Delta s
\end{equation}
其中$T$是相变发生的温度而$\Delta s$是两种相的摩尔熵的差。
或者潜热可以写成两种相摩尔焓的差
\begin{equation}
\label{equ9.3}
\ell=\Delta h
\end{equation}
这可以直接从恒等式$h=Ts+\mu$
(以及摩尔Gibbs函数$\mu$在两种相中都相等这一事实)中得出。
对于很多情况每一种相的摩尔焓都被做成了表。

如果相变发生在气相和液相之间,潜热被称为{\it 汽化热},
而如果发生在气相和固相之间,则被称为{\it 升华热}。

在一个大气压下水的气液相变(沸腾)发生在$373.15K$,
而汽化潜热是$40.7kJ/mole$($540cal/g$)。

在每一种情况下系统从低温相变为高温相都需要{\it 吸收}相应的潜热。
高温相的摩尔熵和摩尔焓都比低温相的要高。

需要注意的是诱使相变发生的手段是不相干的——潜热与之无关。
除了在确定的压强下加热冰(“水平地”穿过图\ref{fig9.1}a中的共存曲线),
也可以在确定的温度下加压(“竖直地”穿过共存曲线)。
两种情况下从热库中提取的潜热是相同的。

实用的水气液共存曲线的形式由“饱和蒸汽表”给出——
名称中的“饱和”代表着蒸汽和液相处于平衡。
(“过热蒸汽表”只编写了蒸汽相的性质。)
来自于Sonntag 和Van Wylen的表9.1给出了这种饱和蒸汽表的一个例子。
每一种相的属性$s,u,v$和$h$都按照惯例列在这些表中;
每一种相变的潜热都是两种相的摩尔焓的差,或者也可以用$T\Delta s$得到。

在热物理数据手册中对于很多种其他的物质都按照这种方式给出了类似的数据。

摩尔体积和摩尔熵与摩尔能量一样,在越过共存曲线的时候是不连续的。
对于水的共存曲线来说这特别有趣。
通常的经验中冰会在液态水中漂浮。
从而固相(冰)的摩尔体积要比液相的摩尔体积{\it 更大}——
这是$H_2O$的一个不平常的特性。
更常见的情况是固相更加致密,拥有更下的摩尔体积。
$H_2O$这个特殊性质的一个平凡的后果就是冰冻水管会胀裂。
而我们在第\ref{sec9.3}中会看到作为一个补偿性的后果就是可以滑冰。
而全部后果中最根本的,是水的这个特殊性质是地球上可以存在生命所必需的。
如果并比液态水更加致密,寒冷的冬天湖和海洋的表面冻结后会沉底;
表面新的液体由于没有了冰层的保护,会继续冻结(并且下沉)
直到所有水都被冻成固态(“下面结冻”而不是“上面结冻”)。

\section{共存曲线的斜率;Clapeyron 方程}
\label{sec9.3}

图\ref{fig9.1}中所示共存曲线并不如直观看起来那么任意;
共存曲线的斜率$dP/dT$完全由两个共存相的性质决定。

共存曲线的斜率物理上是很有趣的。
考虑一个立方体的冰在一玻璃杯的水中处于平衡。
给定外界的压强,那么这个混合系统的温度就由水的固液共存曲线决定了;
如果温度并不在共存曲线上,那要么有些冰会熔化,要么有些水会凝固,
知道温度再次回到共存曲线上(或者其中一个相会消失)。
在一个大气压下,温度会是$273.15K$。
如果外界气压下降——可能由于高度改变导致(这杯水是由一架飞机上的空乘提供的),
或者由于大气条件的改变(风暴的接近)
——于是这杯水的温度会恰当地调整到共存曲线上一个新的点上。
如果压强的改变是$\Delta P$那么温度的改变就是
$\Delta T=\Delta P/(dP/dT)_{cc}$,
其中分母上的导数是共存曲线的斜率。

我们前面提及的滑冰提供了另一个有趣的例子。
冰刀对下面冰的压力会使得冰越过固液共存曲线(图\ref{fig9.2}中竖直向上),
在滑过的表面上共一个液体的润滑层。

滑冰之所可行是由于水的固液共存曲线斜率是负的。
冰会存在于湖的上表面而不是在底部,反应了水的固相的摩尔体积要比液相的大。
这两个并不互相独立的事实的关联,隐含在现在我们要关注的Clapeyron方程中。

考虑如图\ref{fig9.7}所示的四个态。
态$A$和$A'$在共存曲线上,但是他们对应着不同的相
(分别对应着左手边的区域和右手边的区域。)
态$B$和$B'$也类似。
压强差$P_B-P_A$(或者等价地,$P_{B'}-P_{A'}$)
被假设为是无穷小量($=dP$),而温度差$T_B-T_A$($=dP$)也类似。
曲线的斜率是$dP/dT$。

相平衡要求
\begin{equation}
\label{equ9.4}
\mu_A=\mu_{A'}
\end{equation}
以及
\begin{equation}
\label{equ9.5}
\mu_B=\mu_{B'}
\end{equation}
因此
\begin{equation}
\label{equ9.6}
\mu_B-\mu_A=\mu_{B'}-\mu_{A'}
\end{equation}
不过我们又有
\begin{equation}
\label{equ9.7}
\mu_B-\mu_A=-sdT+vdP
\end{equation}
和
\begin{equation}
\label{equ9.8}
\mu_{B'}-\mu_{A'}=-s'dT+v'dP
\end{equation}
其中$s$和$s'$是两个相的摩尔熵而$v$和$v'$是它们的摩尔体积。
通过把方程\eqref{equ9.7}和\eqref{equ9.8}代入方程\eqref{equ9.6}
并重新组合,我们很容易发现
\begin{equation}
\label{equ9.9}
\frac{dP}{dT}=\frac{s'-s}{v'-v}
\end{equation}
\begin{equation}
\label{equ9.10}
\frac{dP}{dT}=\frac{\Delta s}{\Delta v}
\end{equation}
其中$\Delta s$和$\Delta v$是相变对应的摩尔熵和摩尔体积的变化。
根据方程\eqref{equ9.2},潜热是
\begin{equation}
\label{equ9.11}
\ell=T\Delta s
\end{equation}
因此
\begin{equation}
\label{equ9.12}
\frac{dP}{dT}=\frac{\ell}{T\Delta v}
\end{equation}
这就是Clapeyron方程。

Clapeyron方程体现了Le Chatelier原理。
考虑固液相变中潜热为正($s_l>s_s$)并且摩尔体积变化为正($v_l>v_s$)的情况。
相曲线的斜率是正的。
于是固定温度下增加压强就会推动系统到一个更加致密(固体)的相(缓解压强增加),
而温度的增加会推动系统到一个熵更大(液体)的相。
相反地,如果$s_l>s_s$但是$v_l<v_s$,共存曲线的斜率就是负的,
因此增加压强(固定温度$T$下)就会推动系统变到液相——依然是一个更加致密的相。

在应用Clapeyron方程的实践问题中,相对气相忽略液相的摩尔体积
($v_g-v_l\simeq v_g$),并且用理想气体方程($vg\simeq RT/P$)
来近似气体的摩尔体积是可以的。
这个“Clapeyron-Clausius 近似”可以适用于这一节后面的问题。

{\bf 例题}

一个截面为矩形的刚体金属棒平放在一块冰上,两端都伸出一点。
棒的宽度是$2mm$,和冰接触的长度是$25cm$。
两个质量均为$M$的物体分别挂在棒的两端。
整个系统处于大气压强下,并且温度保持在$T=-2^\circ C$。
那么如果想要让金属棒通过熔化后再重新结冰的方式通过冰块,
所需要的$M$的最小值是多少?
给出的数据有水的凝固潜热是$80cal/g$,水的密度是$1g/cm^3$,
以及冰块浮在水面上的时候$\simeq 4/5$的体积淹没在水面下。

{\bf 解答}

Clapeyron方程允许我们找出在$T=-2^\circ C$的时候发生固液相变的压强。
不过我们必须首先用“冰立方数据”来得到固相和液相的摩尔体积差$\Delta v$。
数据给出冰的密度是$0.8g/cm^3$。
更进一步有$\nu_{liq}\simeq 18cm^3/mole$,
以及$\nu_{solid}\simeq 22.5 \times 10^{-6} m^3/mole$。
从而
\begin{equation*}
\left.\frac{dP}{dT}\right)_{cc}=\frac{\ell}{T\Delta v}
=\frac{80\times4.2\times18J/mole}{271\times(-4.5\times10^{-6}Km^3/mole}
=-5\times10^6Pa/K
\end{equation*}
所以需要的压强差就是
\begin{equation*}
P\simeq-5\times10^6\times(-2)\simeq10^7Pa
\end{equation*}
这个压强是由$2Mg$的重量作用在面积$A=5\times10^{-5}m^2$上得到的,
\begin{equation*}
\begin{aligned}
M&=&\frac{1}{2}\Delta P\frac{A}{8}\\
&=&\frac{1}{2}(10^7Pa)(\SI{5e-5}{\meter\square})/\left(9.8\frac{m}{s^2}\right)=2.6Kg
\end{aligned}
\end{equation*}


\section{不稳定的等温线和一级相变}
\label{sec9.4}

我们讨论对一级相变的起因的讨论之前理所应当地关注了Gibbs势的多个极小值。
不过尽管Gibbs势可能是此处的基本对象,
一个等温线的形状是一个更常用的对热力学系统的描述方式。
对于许多种气体,等温线的形状是被van der Waals状态方程
(回溯第\ref{sec3.5}节)描述得很好的(至少半定量的)
\begin{equation}
\label{equ9.13}
P=\frac{RT}{(v-b)}-\frac{a}{v^2}
\end{equation}

这种van der Waals等温线的形状如图\ref{fig9.8}中的$P-v$图所示。

正如在第\ref{sec3.5}中指出的,
van der Waals状态方程可以被看成通过曲线拟合,
基于可信的启发式推理的推断,或者通过基于简单的分子模型的统计力学计算
而得到的“潜在的状态方程”。
也存在其它的经验或者半经验的状态方程,
而且他们的等温线也和图\ref{fig9.8}所示类似。

我们现在来探索一般形式的等温线显示和定义一个相变的方式。

需要立刻注意的是图\ref{fig9.8}中的等温线并不满足固有的处处稳定的条件,
这些条件之一(方程\eqref{equ8.21})是$\kappa_T>0$,或者说
\begin{equation}
\label{equ9.14}
\left(\frac{\partial P}{\partial V}\right)_T<0
\end{equation}
这个条件在一个典型的等温线的$FKM$部分是被破坏的
(为了更加清晰,单独放在图\ref{fig9.9}中展示)。
由于稳定条件被破坏,等温线的这部分一定是非物理的,
因此会在接下来探究的一种方式中被相变代替。

摩尔Gibbs势完全由等温线的形状所决定。
从Gibbs-Duhem关系我们回想起
\begin{equation}
\label{equ9.15}
d\mu=-sdT+vdP
\end{equation}
由此在固定温度下积分
\begin{equation}
\label{equ9.16}
\mu=\int vdP+\phi(T)
\end{equation}
其中$\phi(T)$是温度的待定函数,作为“积分常数”出现。
在固定温度下的被积函数$v(P)$由图\ref{fig9.9}给出,
这里用了最方便的将$P$作为横坐标$v$作为纵坐标的表示方式。
通过任意设定点$A$处的化学势,
我们可以从方程\ref{equ9.16}计算同一条等温线上其它任意一点,
比如$B$处的$\mu$的值
\begin{equation}
\label{equ9.17}
\mu_A-\mu_B=\int_A^Bv(P)dP
\end{equation}
用这种方式我们可以得到图\ref{fig9.10}。
这个表示出了$\mu$关于$P$的图,
可以被当成图\ref{fig9.11}所示$\mu$关于$P$和$T$的关系的三维图的平面截面。
图中画出的$\mu$曲面的四个不同的固定温度的截面对应着四个等温曲线。
还需要注意的是由于$v(P)$在$P$处会取三个值(见图\ref{fig9.8})
而引起的$\mu$关于$P$的曲线中的闭合回路,
会和图\ref{fig9.8}所示一样在高温的时候消失。

最后我们要注意的是由于化学势$\mu$是每一摩尔的Gibbs函数,
所以关系$\mu=\mu(T,P)$构成了一个对于一摩尔物质的基本关系。
于是从图\ref{fig9.11}可以看出我们已经几乎成功地
从单独一个给定的状态方程出发构造出了基本方程,
但是需要回想的是尽管$\mu$-曲面上的每一个轨迹
(在图\ref{fig9.9}中的不同的固定温度的平面上)都有恰当的形式,
但是每个都包含的额外“常数”$\phi(T)$在从一个温度平面变到另一个的时候会消失。
因此尽管我们肯定可以构造一个关于它的基本图谱性质的非常好的图像,
但我们并不知道$\mu(T,P)$曲面的完整形式。

根据这个由van der Waals方程暗示的基本关系的定性图像,
我们回到了稳定性的问题。

考虑一个处于图\ref{fig9.9}中的状态$A$的跟热库和蓄压器接触的系统。
假设蓄压器的压强在保持温度固定的时候准静态地增加。
系统沿着图\ref{fig9.9}中的等温线从点$A$到点$B$的方向行进。
对于小于$P_B$的压强我们看到系统的体积(在给定的压强和温度下)是单值唯一的。
在压强增加到$P_B$以上的时候,三个具有相同$P$和$T$的态都可能是系统所处的态,
比如被记为$C,L$和$N$的态。
这三个态中的$L$是不稳定的,但是在$C$和$N$Gibbs势都是(局域的)极小值。
这两个Gibbs势(或$\mu$)的局域极小值在图\ref{fig9.10}中用$C$和$N$标出。
系统实际选择了态$C$还是态$N$决定于
这两个Gibbs势的局域极小值哪一个更低也就是最小。
图\ref{fig9.10}清楚显示了态$C$是这个压强和温度下真正的物理态。

如果压强继续缓慢增加,就会到达唯一的点$D$。
在这个点$\mu$曲面会如图\ref{fig9.10}所示自相交,
于是$\mu$或者$G$的最小值会出现在曲线的另一个分支上。
因此在比$P_D$更高的压强$P_E=P_Q$下,物理态是$Q$。
在$P_D$以下图\ref{9.9}中等温线的右侧分支是有物理意义的分支,
而在$P_D$以上左侧分支是有物理意义的分支。
{\it 从而从图\ref{fig9.9}中假设的等温线
就可以约化出图\ref{fig9.12}所示的物理的等温线。}

图\ref{fig9.9}中的等温线属于“潜在的基本关系”;
而图\ref{fig9.12}中的等温线是稳定的“热力学基本关系”。

点$D$和$O$是有条件$\mu_D=\mu_O$所决定的,或者从方程\eqref{equ9.17}得到
\begin{equation}
\label{equ9.18}
\int_D^Ov(P)dP=0
\end{equation}
其中这个积分是沿着假设的等温线做的。
针对tu\ref{fig9.9}我们看到这个条件可以
通过把积分分成几部分给出一个直观的图像表示
\begin{equation}
\label{equ9.19}
\int_D^FvdP+\int_F^KvdP\int_K^MvdP\int_M^OvdP=0
\end{equation}
然后按照下面的方式重新排列
\begin{equation}
\label{equ9.20}
\int_D^FvdP-\int_K^FvdP=\int_M^KvdP-\int_M^OvdP
\end{equation}
现在积分$int_D^FvdP$是图\ref{fig9.12}中弧线$DF$下的面积
而积分$int_K^FvdP$是图\ref{fig9.12}中弧线$KF$下的面积。
而这两个积分的差是闭合区域$DFKD$的面积,
也就是图\ref{fig9.12}中标记的区域$I$的面积。
类似的,方程\eqref{equ9.20}右手边表示了图\ref{fig9.12}中$II$的面积,
从而唯一的点$O$和$D$由下面的图形条件所决定
\begin{equation}
\label{equ9.21}
area~I= area~II
\end{equation}
{\it 一条名义上(非单调)的等温线只有在被这个等面积构造
截断以后才能表示一个物理的等温线。}

相变中不仅摩尔体积会有非零的变化,摩尔能量和摩尔熵也会有相应的非零的变化。
熵的改变可以通过沿着假设的等温线$OMKFD$积分这个量
\begin{equation}
\label{equ9.22}
ds=\left(\frac{\partial s}{\partial v}\right)_Tdv
\end{equation}
而算出。
或者根据一个热力学识记图,我们可以写出
\begin{equation}
\label{equ9.23}
\Delta s=s_D-s_O=
\int_{OMKFD}\left(\frac{\partial P}{\partial T}\right)_vdv
\end{equation}
这个熵的差别的几何表述是图\ref{fig9.13}所示相邻等温线的面积差。

由于系统是在固定的压强和温度下从纯粹的相$O$变到相$D$的,
因此它会吸收每摩尔等于$l_{DO}=T\Delta s$的热量。
每摩尔的体积变化是$\Delta v=v_D-v_O$,
而这对应着等于$P\Delta v$的做功。
因此完整的摩尔能量的变化是
\begin{equation}
\label{equ9.24}
\Delta u=u_D-u_O=T\Delta s-P\Delta v
\end{equation}

每一条等温线,比如图\ref{fig.12}中的那一条,现在都被分成了三个区域。
区域$SO$处于液相。
区域$DA$处于气相。
平直的区域$OKD$对应于两种相的混合。
因此整个$P-v$平面可以如图\ref{fig9.14}所示根据相来分类。
气相共存混合区由反向的连接着每条等温线的平直区域的末端的
类似抛物线的曲线所限制。

在这个两相区域中给定的任意一点都代表着
处于通过这一点的等温线的平坦区域末端的两种态的混合。
系统中存在的两种相的比例由“杠杆原理”所决定。
我们假设等温线平直区域两个末端的摩尔体积是$v_\ell$和$v_g$
(需要明确这里认为两种相是液相和气相不是必需的)。
令混合态的摩尔体积是$v=V/N$。
从而如果$x_\ell$和$x_g$是两种相的摩尔比例
\begin{equation}
\label{equ9.25}
V=Nv=Nx_\ell v_\ell+Nx_gv_g
\end{equation}
从中很容易发现
\begin{equation}
\label{equ9.26}
x_\ell=\frac{v_g-v}{v_g-v_\ell}
\end{equation}
以及
\begin{equation}
\label{equ9.27}
x_g=\frac{v-v_\ell}{v_g-v_\ell}
\end{equation}
也就是说,等温线的平直区域上的一个中间点所要求的每一种相的摩尔比例
等于点到平直区域的{\it 相对的}末端的距离的比例。
因此图\ref{fig9.14}中的点$Z$代表了一个液相的摩尔比例
等于$ZD$的“长度”除以$OD$的“长度”的气液混合系统。
这就是非常简便和图像化的杠杆原理。

两相区域的顶点,或者图\ref{fig9.14}中$Q^{\prime\prime}$
和$D^{\prime\prime}$重合的点,
对应着{\it 临界点}——图\ref{fig9.1}a中气液共存曲线终止的点。
温度高于临界温度的等温线是单调的(图\ref{fig9.14})
而摩尔Gibbs势也不会再自相交(图\ref{fig9.10}).

就像$P-v$图存在一个对应着摩尔体积不连续的两相区域,
$T-s$图也存在一个摩尔熵不连续的两相区域。

{\bf 例题 1}

对一个用van der Waals状态方程描述的系统,
找出相应的临界温度$T_{cr}$和临界压强$P_{cr}$。
用约化的变量$\tilde{T}\equiv T/T_{cr},\tilde{P}\equiv P/P_{cr}$
和$\tilde{v}\equiv v/v_{cr}$写出van der Waals 方程。


{\bf 解答}

临界态对应着等温线的水平拐点,也就是
\begin{equation*}
\left(\frac{\partial P}{\partial v}\right)_{T_{cr}}
=\left(\frac{\partial^2 P}{\partial v^2}\right)_{T_{cr}}=0
\end{equation*}
(为什么呢?)联立求解这两个方程得到
\begin{equation*}
v_{cr}=3b~~~P_{cr}=\frac{a}{27b^2}~~~RT_{cr}=\frac{8a}{27b}
\end{equation*}
从而我们可以用约化变量写出van der Waals方程
\begin{equation*}
\tilde{P}=\frac{8\tilde{T}}{3\tilde{v}-1}-\frac{3}{\tilde{v}^2}
\end{equation*}


{\bf 例题 2}

对一个用van der Waals状态方程描述的系统
计算在$P-T$平面上的两相区域的边界的泛函形式。

{\bf 解答}

我们使用前一个例子中定义的约化变量。
考虑在一个固定的温度,并在对应的等温线上应用Gibbs等面积构造。
令对应着约化温度为$\tilde{T}$的两相区域的极值
是$\tilde{v}_g$和$\tilde{v}_\ell$。
对应着方程\eqref{9.20}和\eqref{equ9.21}的等面积构造是
\begin{equation*}
\int_{v_\ell}^{v_g}\tilde{P}d\tilde{v}=\tilde{P_\ell}
(\tilde{v}_g-\tilde{v}_\ell)
\end{equation*}
其中$\tilde{P}_\ell=\tilde{P}_g$是相变发生的时候
(在给定的约化温度$\tilde{T}$下)的约化压强。
读者需要画出等温线,明确前面的方程每一边的意义,
并且使这个陈述的形式与方程\eqref{9.20}和\eqref{equ9.21}相符合;
读者还应该证明方程中使用约化变量的合法性。
对于积分的直接计算给出
\begin{equation*}
\ln(3\tilde{v}_g-1)+\frac{9}{4\tilde{T}}\frac{1}{\tilde{v}_g}
-\frac{1}{3\tilde{v}_g-1}=\ln(3\tilde{v}_\ell-1)
+\frac{9}{4\tilde{T}}\frac{1}{\tilde{v}_\ell}-\frac{1}{3\tilde{v}_\ell-1}
\end{equation*}
联立这个方程和对于$\tilde v_g(\tilde{P},\tilde{P})$以及
$\tilde v_\ell(\tilde{P},\tilde{P})$的van der Waals方程可以给出
每个$\tilde{T}$的下的$\tilde{v}_g$,$\tilde{v}_\ell$和$\tilde{P}$。

\section{一级相变的普遍属性}
\label{sec9.5}

我们对一级相变的讨论是基于能用van der Walls等温线描述的一般形状的等温线。
这个问题可以从更一般的基于热力学势的凹凸性的观点来看。

考虑一个一般的热力学势$U[P_s,\ldots,P_l]$,
这是一个关于$S,X_1,X_2,\ldots,X_{s-1},P_s,\ldots,P_l$的函数。
稳定性的标准是$U[P_s,\ldots,P_l]$必须是一个关于它的广延量的凸函数
并且是一个关于它的强度量的凹函数。
集合上,函数必须在子空间$X_1,\ldots,X_{s-1}$的切平面的上方
并且在子空间$P_s,\ldots,P_l$的下方。

考虑函数$U[P_s,\ldots,P_l]$作为$X_j$的函数,
并且假设它具有如图\ref{9.16}a的形式。
图中还画出了切线$DO$。
需要注意的是函数处于这条切线的上方。
它同时也处于$D$点左侧或者$O$点右侧的点的所有切线的上方。
函数并不会处于$D$和$O$之间的点的切线的上方。
势的局域的曲率对于除了$F$和$M$之间的点以外的所有点都是正的。
不过从$D$到$O$的相之间会有一个相变发生。
局域的曲率在$F$失效前,整体的曲率会在$D$失效(变成负的)。

“修正的”热力学势$U[P_s,\ldots,P_l]$由图\ref{fig9.16}a
(译注:这里原文是9.15,应该是笔误了。)
中的$AD$部分,两相的直线部分$DO$和原来的$OR$部分构成。

直线部分上的一个中间点,比如$Z$,对应着一个$D$和$O$的混合相。
$D$相的摩尔比例随着$Z$点从$D$点移动到$O$点,会线性地从一变到零,
也就是说它满足
\begin{equation*}
X=\frac{(X_j^O-X_j^Z)}{(X_j^O-X_j^D)}
\end{equation*}
这又回到了“杠杆原理”。

处于混合态(例如在$Z$的态)的热力学势$U[P_s,\ldots,P_l]$的值
显然要比处于纯态(在对应着$X_j^Z$的初始曲线上)要小。
因此由构造出的直线给出的混合态确实使得$U[P_s,\ldots,P_l]$最小
并且确实对应着系统的物理的平衡态。

$U[P_s,\ldots,P_l]$对于强度量$P_s$的依赖可适用现在已经熟悉的类似的讨论。
Gibbs势$U[T,P]=N\mu(T,P)$是前面一节研究过的特殊例子。
除了$MF$部分(图\ref{fig9.16}b)以外局域的曲率都是负的。
但是$MD$部分处在画出的$ADF$部分的$D$处切线的上面而不是下面。
只有曲线$ADOR$处处位于切线的下方,因此满足全局稳定性的条件。

因此前一节的特殊结论是适用于所有热力学势的一般性结论。

\section{多组分系统一级相变——Gibbs 相律}
\label{sec9.6}

如果一个系统像水一样(参见图\ref{fig9.1})有多于两种相,相图会变得很复杂。
在多组分系统中两相图被多维空间代替,并且可能的复杂度会迅速增大。
幸运的是尽管如此,允许的复杂度被"Gibbs 相律"严格限制了。
这个对于相的边界形式的稳定性限制同时适用于单组分系统和多组分系统,
不过直接在一般情况下研究它会更方便。

在第\ref{chap8}章中发展的稳定条件,同时适用于多组分系统和单组分系统。
这需要我们只是把不同组分的摩尔数当成完全类似体积$V$和熵$S$的广延量来考虑。
特别的,对于单组份系统,基本关系的形式是
\begin{equation}
\label{equ9.28}
U=U(S,V,N)
\end{equation}
或者用摩尔形式
\begin{equation}
\label{equ9.29}
u=u(s,v)
\end{equation}
对于一个多组分系统基本关系是
\begin{equation}
\label{equ9.30}
U=U(S,V,N_1,N_2,\ldots,N_r)
\end{equation}
而摩尔形式是
\begin{equation}
\label{equ9.31}
u=u(s,v,x_1,x_2,\ldots,x_{r-1})
\end{equation}
摩尔比例$x_j=N_j/N$的总和是1,因此只有$r-1$个$X_j$是独立的,
从而方程\eqref{equ9.31}中只出现了$r-1$个摩尔比例作为独立参数。
这些都是(或者说都应该是)熟悉的,但是这里重复一遍是为了强调这个形式
关于变量$s,v,x_1,x_2,\ldots,x_{r-1}$是完全对称的,
并且稳定条件可以被相应地表述。
在平衡态的时候,能量、焓、Helmholtz和Gibbs势是关于摩尔比例
$x_1,x_2,\ldots,x_{r-1})$的凸函数(参考问题 9.6-1 和9.6-2)
(译注:不知道这里需不需要使用ref,以及用的话格式怎么写?)

如果一个多组分系统不满足稳定条件,相变也会发生。
摩尔比例类似摩尔熵和摩尔体积,在每个相的时候都不同。
因此在总组成中的相一般是不同的。
盐(氯化钠 $NaCl$)和水的混合物在达到沸腾温度的时候会经历相变,
此时气态相几乎是纯水,而共存的液相包含两种成分。
这种情况下两种不同相之间成分的差异是蒸馏提纯的基础。

考虑到相变确实会发生这一事实,不论在单组分或者多组分系统中,
我们都要面对这类多相系统在热力学理论的框架中要如何处理这一难题。
答案实际上很简单,我们只需要考虑把每一种相当成一个简单系统,
而给定的系统当做一个复合系统。
这样简单系统或者相之间的“墙”也就成了完全无约束的,
从而也就可以用适用于无约束墙的方法来分析。

例如考虑一个保持在温度$T$压强$P$并且包含两种组分的混合物的容器。
可以观察到系统会包含两种相:一个液相和一个固相。
我们希望找出每一种相的构成。

第一种组分液相的化学势是$\mu_1^{(L)}(T,P,x_1^{(L)})$,
而固相则是$\mu_1^{(S)}(T,P,x_1^{(S)})$;
需要注意的是对于每一种相$\mu_1$的函数形式是不同的。
对应第一种组分从一个相变到另一个相的平衡条件是
\begin{equation}
\label{equ9.32}
\mu_1^{(L)}(T,P,x_1^{(L)})=\mu_1^{(S)}(T,P,x_1^{(S)})
\end{equation}

类似的,第二种组分的化学势是$\mu_2^{(L)}(T,P,x_1^{(L)})$和
$\mu_2^{(S)}(T,P,x_1^{(S)})$;我们可以用$x_1$而不是$x_2$来表示这些
是因为$x_1+x_2$对于每一个相都是一。
从而令$\mu_2^{(L}$和$\mu_2^{(S)}$相等给出了第二个方程,
并且与方程\eqref{equ9.32}一起决定了$x_1^{(L)}$和$x_1^{(S)}$。

让我们假设在前述系统中观察到了三个共存相。
将这些用$I$,$II$,和$III$来标记,我们得到对第一种组分
\begin{equation}
\label{equ9.33}
\mu_1^{I}(T,P,x_1^{I})=\mu_1^{II}(T,P,x_1^{II})=\mu_1^{III}(T,P,x_1^{III})
\end{equation}
并且对于第二种组分也有一对类似的方程。
从而我们有四个方程并且只有三个组分变量:
$x_1^I$,$x_1^{II}$,和$x_1^{III}$。
这意味着我们不能先验地同时自由选择$T$和$P$,但如果给定了$T$,
那么四个方程就决定了$P$,$x_1^I$,$x_1^{II}$,和$x_1^{III}$。
尽管可以任意选择一个温度和一个压强然后得到一个两种相的态,
但是在一个特定压强下三相态只有在特定的温度下才能存在。

在同一个系统中,我们可以探究四相共存态是否存在。
类似方程\eqref{equ9.33},对于第一个组分我们有三个方程,对第二个也有三个。
从而我们得到了包含$T$,$P$,$x_1^I$,$x_1^{II}$,$x_1^{III}$,和$x_1^{IV}$的六个方程。
这表示我们只有在唯一确定的温度和压强下才有四相共存,
每一个都不能由实验者预选,而只能由系统的性质唯一确定。

两组分系统中不可能五相共存,因为八个方程对于七个变量
$(T,P,x_1^I,\ldots,x_1^V)$来说是超定的,
并且一般是不可能有解的。

对于多组分多相系统,我们可以很容易地重复前述对变量的计数。
对于一个有$r$种组分的系统,第一个相的化学势是变量
$T,P,x_1^I,x_2^I,\ldots,x_{r-1}^I$的函数。
第二个相的化学势是$T,P,x_1^{II},x_2^{II},\ldots,x_{r-1}^{II}$的函数。
如果有$M$个相,独立变量的完整集合包含了$T$,$P$,和$M(r-1)$个摩尔比例;
也就是总共$2+M(r-1)$个变量。
对于每一个组分都有$M-1$个化学势相等的方程,或者说总共$r(M-1)$个方程。
从而可以任意给定的变量的数$f$就是$[2+M(r-1)]-r(M-1)$,或者
\begin{equation}
\label{equ9.34}
f=r-M+2
\end{equation}
在一个$r$种组分$M$个相的系统的变量集
$(T,P,x_1^I,x_2^I,\ldots,x_{r-1}^M)$中有$r-M+2$个可以任意给出
这个事实就是{\it Gibbs 相律}

$f$这个量也可以被诠释为前面在\ref{sec3.2}节中介绍
并被定义成可当成独立变量的{\it 强度量}的数目的{\it 热力学自由度}的数目。
为了证明这一诠释,我们现在用一种直接的方法来数热力学自由度,
并显示出这与方程\eqref{equ9.34}相符。

一个单组分单相系统有两个自由度,
Gibbs-Duhem 关系可以从三个变量$T,P,\mu$中消去一个。
一个单组分两相系统有三个强度量
($T$,$P$,和$\mu$,每一个对于不同的相都是相同的)
以及两个Gibbs-Duhem 关系。
从而就只有一个自由度。
在图\ref{fig9.1}中每组相都在相应的一维的区域(曲线)上共存。

如果对于一个单组分系统有三种共存相,
三个Gibbs-Duhem关系就完全觉得了三个强度量$T$,$P$,和$\mu$。
三相只能在一个唯一的零维区域,或者说一个点共存;
也就是图\ref{fig9.1}中的几个“三相点”。

对于多组分多相系统自由度可以用类似的方式简单地计算。
如果系统有$r$个组分,就存在$r+2$个强度量:
$T,P,\mu_1,\mu_2,\ldots,\mu_r$。
每一个参数对于不同的相都是相同的。
但是对于$M$个相中的每一个都有一个Gibbs-Duhem关系。
这$M$个关系就把独立变量的数目减少到$(r+2)-M$。
从而自由度的数目$f$就是$r-M+2$,跟方程\eqref{9.34}给出的一样。

从而Gibbs相律可以陈述如下。
{\it 在一个有$r$种组分和$M$种共存相的系统中,
可以从集合$(T,P,x_1^I,x_2^I,\ldots,x_{r-1}^M)$
或者集合$T,P,\mu_1,\mu_2,\ldots,\mu_r$中任意选定$r-M+2$个变量。}

现在证实Gibbs相律对于单组分和两组分系统给出的结果
和我们前几段发现的结果相同就是一个非常简单的事情了。
对于单组分系统$r=1$,从而如果$M=3$就有$f=0$。
这和我们前面关于单组分系统三相点唯一的总结是相符的。
类似的,对于两组分系统我们可以看到四相共存是唯一的一点($f=0,r=2,M=4$),
而三相系统温度可以任选($f=1,r=2,M=3$),
两相系统则是$T$和$P$都可以任选($f=2,r=2,M=2$)。

\section{两组分系统的相图}
\label{sec9.7}

Gibbs相律(方程\eqref{equ9.34})为对相图假定的可能形式的研究提供了基础。
这些相图,特别是关于二元(两种组分)和三元(三种组分)系统的,
在冶金学和物理化学中有着特别重要的意义,
而且在关于它们的分类上已经做了很多工作。
为了阐明相律的应用,我们将会讨论二元系统的两种典型相图。

对于单组分系统每摩尔的Gibbs函数如同图\ref{fig9.11}
中的三维表示,是关于温度和压强的函数。
两维的$T-P$平面上的“相图”(比如图\ref{fig9.1})
是($\mu$曲面和自己的)相交曲线在$T-P$平面上的投影。

对于二元系统摩尔Gibbs函数$G/(N_1+N_2)$
是关于{\it 三个}变量$T,P$和$x_1$的函数。
跟图\ref{fig9.11}类似的也就成了四维的,从而跟$T-P$相图类似的是三维的。
它是通过将相交的“超曲面”投影到$P,T,x_1$“超平面”上得到的。

对于一个简单但是常见类型的二元气液系统的三维相图如图\ref{fig9.17}所示。
显然由于为了画图方便的原因,三维空间通过一组二维的常压截面来表示。
在一个固定的摩尔比例$x_1$和固定压强的时候气相在高温时稳定而液相在低温时稳定。
在某个温度,比如图\ref{fig9.17}中标为$C$的时候,系统分成了两种相
——在$A$的液相和在$B$的气相。
图\ref{fig9.17}中点$C$处的构成类似于图\ref{fig9.14}中的点$Z$
类似杠杆定律的形式显然也是适用的。

图\ref{fig9.17}中标记为“气体”的部分是一个三维区域,
(译注,这里的气体不知道是否应该保留为gas,因为我们不修图,所以图上会是英文)
而$T,P$和$x_1$在这个区域可以独立变化。
这对于标记为“液体”的区域也是成立的。
两种情况都有$r=2,M=1$以及$f=3$。

图\ref{fig9.17}中点$C$表示的态是一个真正的两相态,由$A$和$B$复合而成。
从而只有$A$和$B$是物理的点,而点$C$占据的阴影区域是图中一种非物理的“洞”。
两相区域是图\ref{fig9.17}中包含着阴影体积的曲面。
这个曲面是二维的($r=2,M=2,f=2$)。
特定的$T$和$P$唯一决定了$x_1^A$和$x_1^B$。

如果一个摩尔比例$x_1^A$的二元液体在大气压下加热,
它会沿着图\ref{fig9.17}中对应的图中的垂直线移动。
当它达到点$A$的时候,会开始沸腾。
逸出的蒸气会拥有对应着点$B$的构成。

对于一个固液二组分系统,一个常见类型的相图在同\ref{fig9.18}中展示出来,
其中只画了一个常数压强的截面。
这里存在着具有不同晶体结构的两个分开的固相:
一个被标为$\alpha$另一个被标为$\beta$。
曲线$BDHA$被称为{\it 液相}曲线,而曲线$BEL$和$ACJ$被称为{\it 固相}曲线。
点$G$对应一个两相系统,一部分处于$H$的液体和一部分处于$F$的固体。
点$K$对应着处于$J$的$\alpha$固体和处于$L$的$\beta$固体。

如果具有组成$x_H$的液体被冷却,首先沉淀的固体具有组成$x_F$。
如果想得到和液体具有相同构成的固体,需要从具有组成$x_D$的液体开始。
具有这种组成的液体被称为{\it 共熔}溶液。
一个共熔溶液会急剧且均匀地凝固,在冶金学实践中制造好的合金铸件。

固相和液相曲线是完整的$T-x_1-P$空间中二维曲面的轨迹。
共熔点$D$是是整个$T-x_1-P$空间曲线的轨迹。
共熔合金是一个三相区域,其中在$D$的液相,在$E$的$\beta$固相,
和在$C$的$\alpha$固相共存。
三相系统可以存在在一个一维曲线上这一事实符合相律($r=2,M=3,f=1$)。

假设我们从一个液相比如$N$出发。
保持$T$和$x_1$不变并减小压强,这样我们就会沿着在$T-x_1-P$空间中
垂直于图\ref{fig9.18}所在平面的直线移动。
最终我们会到达一个代表气液相变的两相曲面。
这个相变会在给定温度和组成下载一个特定的压强发生。
类似的,对应点$Q$的温度和组成存在另一个特定的压强
使得$\beta$固相和它的蒸气保持平衡。
对于每一个点$T,x_1$,我们都可以用这种方式得到一个特定的压强$P$。
于是可以画出如图\ref{fig9.19}的相图。
这个相同与图\ref{fig9.18}不同点在于每一点的亚青是不同的,
并且每一点都代表一个至少有两相的系统(其中一个相是蒸气)。
曲线$B^\prime D^\prime$现在是一个一维曲线(M=3,f=1),
而共熔点$D^\prime$是一个唯一的点($M=4,f=0$)。
点$B^\prime$是纯粹第一种组分的三相点,
而点$A^\prime$是纯粹第二种组分的三相点。

尽管图\ref{fig9.18}和\ref{fig9.19}外观上看起来非常类似,
它们的意义却显然是非常不同的,
同时如果不能仔细区分这两种相图的话也会很容易导致困惑。
具体形式的相图可以在细节上有无数种差异,
但是各种多相区域相交的维数是完全由相律决定的。